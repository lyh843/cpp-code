\documentclass{article}
\usepackage{amsthm}
\usepackage{graphicx}
\usepackage{ctex}
\usepackage{amsmath}
\usepackage{amssymb} % 添加 amssymb 以支持更多符号
\usepackage{amsfonts}
\usepackage{listings}
\usepackage{xcolor} % 颜色支持
\definecolor{ForestGreen}{RGB}{34, 139, 34}
\lstset{
    language=C++, % 设置语言
    basicstyle=\ttfamily\small, % 基本字体
    numbers=left, % 行号在左侧
    numberstyle=\tiny\color{gray}, % 行号样式
    frame=single, % 边框
    captionpos=b, % 标题位置(底部)
    breaklines=true, % 自动换行
    keywordstyle=\color{blue}, % 关键字颜色
    commentstyle=\color{ForestGreen}, % 注释颜色
    stringstyle=\color{red}, % 字符串颜色
    tabsize=4 % 制表符替换为4空格
}
\title{高级程序设计 作业一}
\author{李云浩 241880324}   
\date{\today}
\begin{document}
\maketitle
\section{题目 1}
"this"指针是在类的各个除静态成员函数的成员函数中隐藏的一个形参,
这个指针指向调用该成员函数的对象。编译时会添加该指针,以此来表面当前操作的对象。
当在函数进行过程中时需要对对象整体进行操作的时候,这可以显式使用"this"指针,
来代替对象。以及当函数的参数与类的数据成员同名时,
可以用"this"指针指向类的数据成员。

\section{题目 2}
\textbf{需要自定义析构函数的情况:}
\begin{itemize}
    \item 1.对象在创建后申请了一些资源,并且在消亡前没有归还。
    例如:对象的数据成员中包含指针,在创建后指针"new"了一个动态内存空间。
    \item 2.程序运行时会需要显式调用析构函数的时候,将一些数据成员还原到初始值。
    例如:用对象模拟栈时,调用析构函数清空栈,需要让$len = 0; \quad str = NULL$。
    \item 2.想在析构时输出“析构成功”等语句。
    例如:
    \begin{lstlisting}
    ~A(){
        cout<<"destroyed success."<<endl;
    }
    \end{lstlisting}
\end{itemize}

\textbf{析构函数需要归还的资源:}
\begin{itemize}
    \item 1.对象在创建后额外申请的资源(动态内存)。
\end{itemize}

\textbf{析构函数不需要归还的资源:}
\begin{itemize}
    \item 1.对象的数据成员。
    \item 2.对象自身。
\end{itemize}

\section{题目 3}
\textbf{调用拷贝构造函数的情况:}
\begin{itemize}
    \item 1.在创建一个对象时,使用另一个已有的同类对象对其进行初始化。
    \item 2.把对象作为值参数传给函数时。
    \item 3.把对象作为函数的返回值时。
\end{itemize}

\textbf{隐式拷贝构造函数可能导致运行错误的原因:}
浅拷贝会将对象的数据成员的值直接复制给新创建的对象。当对象的成员中存在指针时,
新对象的指针会与所拷贝的对象的指针指向同一个地址,而不是把指向的内容拷贝到新的内存中。
那么此后通过新对象修改该空间的内容时,会影响一开始的对象。此外,在析构函数中释放动态内存时,
会对同一块内存空间进行两次归还,导致程序运行错误。或者其中一个对象析构时
会使得另一个对象无法调用该内存空间的数据。

\section{题目 4}
浅拷贝会把对象的数据成员的值直接复制给新创建的对象。当对象的成员中存在指针时,
浅拷贝会把指针所指向的地址拷贝一份给新对象,而深拷贝会把指针所指向的内存空间的内容进行拷贝
让新对象指向这块新拷贝出来的空间。

\section{题目 5}
\begin{lstlisting}
    #include <iostream>

class Account{
public:
    //构造函数,针对用户传入的参数对对象进行赋值
    Account(char* a, char* b, int c);
    //析构函数,释放用户姓名和账号所申请的空间
    ~Account();
    //显示函数,无需对对象的数值进行修改
    void display() const;    
    //存入函数
    void deposit(int in_money);    
    //取出函数
    void take_out(int out_money);   
private:
//账户的数据成员,其中姓名和账号用char*类型指向用户输入的字符串。
    char *name; //储户姓名
    char *account_number;   //账号
    int money;  //存款
};
\end{lstlisting}

\section{题目 6}
\begin{lstlisting}
    class Deque{
    public:
        Deque();
        ~Deque();
        void push_front(int x);
        void push_back(int x);
        void pop_front();
        void pop_back();
        size_t size() const;
    private:
        struct Node{
            int val;
            struct Node* prev;
            struct Node* next;
            Node(int x);
        };
        Node* head;
        Node* tail;
    };

    Deque::Deque(){
        head = nullptr;
        tail = nullptr;
    }

    Deque::Node::Node(int x): val(x), prev(nullptr), next(nullptr){}

    Deque::~Deque(){
        while(head)
        {
            Node *denode = head;
            head = head->next;
            delete denode;
        }
        tail = nullptr;
    }

    void Deque::push_front(int x){
        Node* newnode = new Node(x);
        if(head)
        {
            newnode->next = head;
            head->prev = newnode;
            head = newnode;
        }
        else
        {
            head = newnode;
            tail = newnode;
        }
    }

    void Deque::push_back(int x){
        Node* newnode = new Node(x);
        if(tail)
        {
            newnode->prev = tail;
            tail->next = newnode;
            tail = newnode;
        }
        else
        {
            head = newnode;
            tail = newnode;
        }
    }

    void Deque::pop_front(){
        if(head)
        {
            Node* denode = head;
            if(head == tail) tail = nullptr;
            head = head->next;
            if(head) head->prev = nullptr;
            delete denode;
        }
    }

    void Deque::pop_back(){
        if(tail)
        {
            Node* denode = tail;
            if(tail == head) head = nullptr;
            tail = tail->prev;
            if(tail) tail->next = nullptr;
            delete denode;
        }
    }

    size_t Deque::size()const{
        Node* curr = head;
        int size = 0;
        while(curr)
        {
            size++;
            curr = curr->next;
        }
        return size;
    }
\end{lstlisting}


\section{题目 7}
\begin{lstlisting}
    #include <iostream>
    #include <cstring>
    using namespace std;
    class Book{
        static int BookCnt;
        char *name;
        public:
            Book(const char * _name);
            ~Book();
            char * get_name() const;
            //void set_name(const char * _name) const;
            //该函数会对对象的数据进行修改,不应该后缀const
            void set_name(const char * _name);
            //对象的数据成员中含有指针,应该添加深拷贝函数
            Book(const Book& other);
    };
    
    //对BookCnt进行初始化
    int Book::BookCnt = 0;
    
    Book::Book(const char * _name){
        name = new char[strlen(_name) + 1];
        strcpy(name, _name);
        BookCnt++;
    }
    
    //增加深拷贝构造函数
    Book::Book(const Book& other){
        name = new char[strlen(other.name) + 1];
        strcpy(name, other.name);
        BookCnt++;
    }
    
    Book::~Book(){
        delete []name;
        name = nullptr;
        //如果确保析构后不会再次调用已被析构的对象,此处应该赠添BookCnt--
        BookCnt--;
    }
    
    char* Book::get_name() const{
        return name;
    }
    
    //void Book::set_name(const char* _name) const{
    //该函数会对对象的数据进行修改,不应该后缀const
    void Book::set_name(const char* _name){
        delete []name;
        name = new char[strlen(_name) + 1];
        strcpy(name, _name);
    }
    
    int main(){
        Book b1("Computer Science");
        //此处会进行拷贝,对象的数据成员中存在指针,应该使用深拷贝
        Book b2(b1);
        return 0;
    }
\end{lstlisting}
\end{document}